%% start of file `template.tex'.
%% Copyright 2006-2013 Xavier Danaux (xdanaux@gmail.com).
%
% This work may be distributed and/or modified under the
% conditions of the LaTeX Project Public License version 1.3c,
% available at http://www.latex-project.org/lppl/.


\documentclass[11pt,a4paper,sans]{moderncv}        % possible options include font size ('10pt', '11pt' and '12pt'), paper size ('a4paper', 'letterpaper', 'a5paper', 'legalpaper', 'executivepaper' and 'landscape') and font family ('sans' and 'roman')

% modern themes
\moderncvstyle{banking}                            % style options are 'casual' (default), 'classic', 'oldstyle' and 'banking'
\moderncvcolor{black}                                % color options 'blue' (default), 'orange', 'green', 'red', 'purple', 'grey' and 'black'
%\renewcommand{\familydefault}{\sfdefault}         % to set the default font; use '\sfdefault' for the default sans serif font, '\rmdefault' for the default roman one, or any tex font name
%\nopagenumbers{}                                  % uncomment to suppress automatic page numbering for CVs longer than one page

% character encoding
\usepackage[utf8]{inputenc}                       % if you are not using xelatex ou lualatex, replace by the encoding you are using
%\usepackage{CJKutf8}                              % if you need to use CJK to typeset your resume in Chinese, Japanese or Korean

% adjust the page margins
\usepackage{geometry}
\geometry{a4paper, portrait, margin=0.5in}
%\setlength{\hintscolumnwidth}{3cm}                % if you want to change the width of the column with the dates
%\setlength{\makecvtitlenamewidth}{10cm}           % for the 'classic' style, if you want to force the width allocated to your name and avoid line breaks. be careful though, the length is normally calculated to avoid any overlap with your personal info; use this at your own typographical risks...

\usepackage{import}

\usepackage{verbatim} %allows multi-line comments

% personal data
\name{Fawaz}{Shah}
%\title{Curriculum Vitae}                               % optional, remove / comment the line if not wanted
\address{4 Cookham Dene Close, Chislehurst, Kent, BR7 5QW}% optional, remove / comment the line if not wanted; the "postcode city" and and "country" arguments can be omitted or provided empty
%\phone[mobile]{+44 7548822935}                   % optional, remove / comment the line if not wanted
%\phone[fixed]{01234 123456}                    % optional, remove / comment the line if not wanted
%\phone[fax]{+3~(456)~789~012}                      % optional, remove / comment the line if not wanted
\email{fs2217@ic.ac.uk}                               % optional, remove / comment the line if not wanted
\homepage{fawazshah.github.io}                         % optional, remove / comment the line if not wanted
\extrainfo{Joint Maths and Computing (JMC) student}                 % optional, remove / comment the line if not wanted
%\photo[64pt][0.4pt]{picture}                       % optional, remove / comment the line if not wanted; '64pt' is the height the picture must be resized to, 0.4pt is the thickness of the frame around it (put it to 0pt for no frame) and 'picture' is the name of the picture file
%\quote{Some quote}                                 % optional, remove / comment the line if not wanted

% to show numerical labels in the bibliography (default is to show no labels); only useful if you make citations in your resume
%\makeatletter
%\renewcommand*{\bibliographyitemlabel}{\@biblabel{\arabic{enumiv}}}
%\makeatother
%\renewcommand*{\bibliographyitemlabel}{[\arabic{enumiv}]}% CONSIDER REPLACING THE ABOVE BY THIS

% bibliography with mutiple entries
%\usepackage{multibib}
%\newcites{book,misc}{{Books},{Others}}
%----------------------------------------------------------------------------------
%            content
%----------------------------------------------------------------------------------
\begin{document}
%\begin{CJK*}{UTF8}{gbsn}                          % to typeset your resume in Chinese using CJK
%-----       resume       ---------------------------------------------------------
\maketitle

\vspace{-30pt}

\section{Education}

\vspace{5pt}

\begin{itemize}

\item{
\cventry
{2017--2020}
{BEng. Mathematics and Computer Science (Joint Honours)}
{Imperial College London}
{}{}{
\noindent
\\
Current modules of interest: Java, Logic, Graphs and Algorithms, Applied Methods and Linear Algebra
}
} % arguments 3 to 6 can be left empty

\item{
\cventry
{2010--2017}
{GCSEs and A Levels}
{St. Olave's Grammar School}
{}{}{
\noindent
\\
A Levels: 4 A*s (Maths, Further Maths, Chemistry, Physics) \& 1 A (EPQ)
\\
AS Levels: 1 A (Computing)
\\
GCSEs: 10 A*s \& 2 As
}}

\end{itemize}

\section{Technical and Personal Skills}

\begin{itemize}

\item Written Python code for maths-intensive personal projects, such as matrix inversion and manipulation of coprime theory and the Mandelbrot set

\vspace{3pt}

\item Implemented Haskell in scenarios such as writing a mail-merge program, generating fractals using Lindenmayer systems and creating a symbolic calculus interpreter

\vspace{3pt}

\item Implemented Java in situations such as solving the shortest path problem, scheduling timetables and basic critical path analysis

\vspace{3pt}

\item{Written a countdown web app in HTML5, using CSS3 for styling and JavaScript for logic-handling}

\vspace{3pt}

\item Hosted and managed a public web server on a Raspberry Pi, using knowledge of Bash and Linux

\vspace{3pt}

\item Familiar with git and programming on Unix systems

\end{itemize}

\section{Positions of Responsibility}

\begin{itemize}

\item \textbf{Imperial College JMC Year 1 Academic Representative} \hfill Sep. 2017 - Jul. 2018

\vspace{3pt}

Current duties include:

\vspace{3pt}

\begin{itemize}

\item Collecting feedback from other first year JMC students and passing it on to the coordinators of the degree

\vspace{3pt}

\item Reporting any academic issues to relevant lecturers or administrative staff

\vspace{3pt}

\item Working with senior members of staff and other representatives to solve problems faced by JMC students

\end{itemize}

\vspace{6pt}

\item \textbf{President of St. Olave’s Computer Science Society} \hfill Jan. 2016 - Dec. 2016

\vspace{3pt}

\begin{itemize}

\item Scheduled regular meetings throughout the year, presented talks on significant topics such as quantum computing and artificial intelligence, and encouraged younger students to give presentations themselves

\vspace{3pt}

\item Expanded the size of the society, by opening it up to all years instead of just 6th form and promoting it heavily

\vspace{3pt}

\item Organized a presentation given by a senior developer at Oracle, achieving record turnout to the Society

\vspace{3pt}

\item Helped organize the St. Olave’s Programming Competition, a 1-term-long coding competition for younger
students

\end{itemize}

\vspace{6pt}

\item \textbf{St. Olave’s Computing Prefect} \hfill Sep. 2016 - Feb. 2017

\begin{itemize}

\vspace{3pt}

\item Mentored a GCSE Computing student, checking up regularly and ensuring he was coping well with the curriculum

\vspace{3pt}

\item Arranged St. Olave’s Raspberry Pi club, helping younger years to learn Bash and Python

\end{itemize}

\end{itemize}

\section{Notable Achievements}

\begin{itemize}

\item Published a dissertation on “The socioeconomic effects of artificial intelligence in the future” in the St. Olave’s Academic Journal

\item Awarded the St. Olave's Senior Colours tie and the H.G. Abel prize for academic excellence

\end{itemize}

% Publications from a BibTeX file without multibib
%  for numerical labels: \renewcommand{\bibliographyitemlabel}{\@biblabel{\arabic{enumiv}}}% CONSIDER MERGING WITH PREAMBLE PART
%  to redefine the heading string ("Publications"): \renewcommand{\refname}{Articles}
\nocite{*}
\bibliographystyle{plain}
\bibliography{publications}                        % 'publications' is the name of a BibTeX file

% Publications from a BibTeX file using the multibib package
%\section{Publications}
%\nocitebook{book1,book2}
%\bibliographystylebook{plain}
%\bibliographybook{publications}                   % 'publications' is the name of a BibTeX file
%\nocitemisc{misc1,misc2,misc3}
%\bibliographystylemisc{plain}
%\bibliographymisc{publications}                   % 'publications' is the name of a BibTeX file

%-----       letter       ---------------------------------------------------------

\end{document}


%% end of file `template.tex'.
