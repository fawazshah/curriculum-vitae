%% start of file `template.tex'.
%% Copyright 2006-2013 Xavier Danaux (xdanaux@gmail.com).
%
% This work may be distributed and/or modified under the
% conditions of the LaTeX Project Public License version 1.3c,
% available at http://www.latex-project.org/lppl/.


\documentclass[11pt,a4paper,sans]{moderncv}        % possible options include font size ('10pt', '11pt' and '12pt'), paper size ('a4paper', 'letterpaper', 'a5paper', 'legalpaper', 'executivepaper' and 'landscape') and font family ('sans' and 'roman')

% modern themes
\moderncvstyle{banking}                            % style options are 'casual' (default), 'classic', 'oldstyle' and 'banking'
\moderncvcolor{black}                                % color options 'blue' (default), 'orange', 'green', 'red', 'purple', 'grey' and 'black'
%\renewcommand{\familydefault}{\sfdefault}         % to set the default font; use '\sfdefault' for the default sans serif font, '\rmdefault' for the default roman one, or any tex font name
%\nopagenumbers{}                                  % uncomment to suppress automatic page numbering for CVs longer than one page

% character encoding
\usepackage[utf8]{inputenc}                       % if you are not using xelatex ou lualatex, replace by the encoding you are using
%\usepackage{CJKutf8}                              % if you need to use CJK to typeset your resume in Chinese, Japanese or Korean

% adjust the page margins
\usepackage{geometry}
\geometry{a4paper, portrait, margin=0.5in}
%\setlength{\hintscolumnwidth}{3cm}                % if you want to change the width of the column with the dates
%\setlength{\makecvtitlenamewidth}{10cm}           % for the 'classic' style, if you want to force the width allocated to your name and avoid line breaks. be careful though, the length is normally calculated to avoid any overlap with your personal info; use this at your own typographical risks...

\usepackage{import}

\usepackage{verbatim} %allows multi-line comments

% personal data
\name{Fawaz}{Shah}
%\title{Curriculum Vitae}                               % optional, remove / comment the line if not wanted
\address{4 Cookham Dene Close, Chislehurst, Kent, BR7 5QW}% optional, remove / comment the line if not wanted; the "postcode city" and and "country" arguments can be omitted or provided empty
%\phone[mobile]{+44 7548822935}                   % optional, remove / comment the line if not wanted
%\phone[fixed]{01234 123456}                    % optional, remove / comment the line if not wanted
%\phone[fax]{+3~(456)~789~012}                      % optional, remove / comment the line if not wanted
\email{fs2217@ic.ac.uk}                               % optional, remove / comment the line if not wanted
\homepage{fawazshah.github.io}                         % optional, remove / comment the line if not wanted
\extrainfo{Joint Maths \& Computing (JMC) student}                 % optional, remove / comment the line if not wanted
%\photo[64pt][0.4pt]{picture}                       % optional, remove / comment the line if not wanted; '64pt' is the height the picture must be resized to, 0.4pt is the thickness of the frame around it (put it to 0pt for no frame) and 'picture' is the name of the picture file
%\quote{Some quote}                                 % optional, remove / comment the line if not wanted

% to show numerical labels in the bibliography (default is to show no labels); only useful if you make citations in your resume
%\makeatletter
%\renewcommand*{\bibliographyitemlabel}{\@biblabel{\arabic{enumiv}}}
%\makeatother
%\renewcommand*{\bibliographyitemlabel}{[\arabic{enumiv}]}% CONSIDER REPLACING THE ABOVE BY THIS

% bibliography with mutiple entries
%\usepackage{multibib}
%\newcites{book,misc}{{Books},{Others}}
%----------------------------------------------------------------------------------
%            content
%----------------------------------------------------------------------------------
\begin{document}
%\begin{CJK*}{UTF8}{gbsn}                          % to typeset your resume in Chinese using CJK
%-----       resume       ---------------------------------------------------------
\maketitle

\vspace{-40pt}

\section{Education}

\vspace{5pt}

\begin{itemize}

\item{
\cventry
{2017--2020}
{BEng. Mathematics and Computer Science (Joint Honours)}
{Imperial College London}
{}{}{
\noindent
\\
On track for First Class Honours (1st year result: 72\%)
}
} % arguments 3 to 6 can be left empty

\item{
\cventry
{2010--2017}
{GCSEs and A Levels}
{St. Olave's Grammar School}
{}{}{
\noindent
\\
A Levels: 4 A*s (Maths, Further Maths, Chemistry, Physics) \& 1 A (EPQ)
\\
AS Levels: 1 A (Computing)
\\
GCSEs: 10 A*s \& 2 As
}}

\end{itemize}

\section{Technical and Personal Skills}

\begin{itemize}

\item Written Python code for mathematical personal projects, such as solving sets of simultaneous equations and manipulation of coprime theory and the Mandelbrot set
\item Implemented Java solutions in situations such as solving the shortest path problem, modelling traffic flow and solving mazes
\item Led my team through a group project in building a binary emulator and an assembler in C
\item Built and managed a public web server on a Raspberry Pi, using a LAMP stack
\item Familiar with Git and programming on Linux systems
\end{itemize}

\section{Work Experience}

\begin{itemize}

\item{
\cventry
{Jul. - Aug. 2018}
{Software Engineer - Intern}
{Netcraft Ltd.}
{}{}{
\begin{itemize}
\item Implemented new features to integrate two of Netcraft's biggest services, C2 and Takedown
\item Wrote production code in Perl and SQL, refactored some of the existing Perl codebase for ease of use
\item Gained experience in full stack development and exposure to the MVC interface pattern
\item Integrated C2 with the older Contracts system, and rewrote old documentation with instructions for integrating with newer systems
\end{itemize}
}}

\item{
\cventry
{Apr. 2018 - present}
{Volunteer Web Developer}
{SolidariTee National}
{}{}{
\begin{itemize}
\item Redesigned the website front-end, using HTML5 and CSS3, to make it more mobile-friendly
\item Gained experience with using npm and ZURB Foundation
\end{itemize}
}}

\end{itemize}

\section{Positions of Responsibility}

\begin{itemize}

\item{
\cventry
{Sep. 2017 - Jul. 2018}
{JMC Year 1 Academic Representative}
{Imperial College London}
{}{}{
\begin{itemize}
\item Liaised with senior members of staff and other representatives to solve problems faced by JMC students
\item Collected feedback from other first year JMC students and passed it on to the coordinators of the degree
\item Reported any academic issues to relevant lecturers or administrative staff
\end{itemize}
}}

\item{
\cventry
{Jan. - Dec. 2016}
{President of Computer Science Society}
{St. Olave's Grammar School}
{}{}{
\begin{itemize}
\item Gave regular presentations throughout the year and encouraged younger students to give presentations themselves
\item Expanded the size of the society, by opening it up to all years instead of just 6th form and promoting it heavily
\item Organized a talk given by a senior developer at Oracle Corporation, achieving record turnout to the Society
\end{itemize}
}}

\end{itemize}

\section{Notable Achievements}

\begin{itemize}

\item Awarded the St. Olave's Senior Colours tie and the H.G. Abel prize for academic excellence

\item Awarded a Merit in Imperial College's Japanese Horizons Level 1 Course

\item Represented Imperial College at Just Bollywood 2017

\end{itemize}

% Publications from a BibTeX file without multibib
%  for numerical labels: \renewcommand{\bibliographyitemlabel}{\@biblabel{\arabic{enumiv}}}% CONSIDER MERGING WITH PREAMBLE PART
%  to redefine the heading string ("Publications"): \renewcommand{\refname}{Articles}
\nocite{*}
\bibliographystyle{plain}
\bibliography{publications}                        % 'publications' is the name of a BibTeX file

% Publications from a BibTeX file using the multibib package
%\section{Publications}
%\nocitebook{book1,book2}
%\bibliographystylebook{plain}
%\bibliographybook{publications}                   % 'publications' is the name of a BibTeX file
%\nocitemisc{misc1,misc2,misc3}
%\bibliographystylemisc{plain}
%\bibliographymisc{publications}                   % 'publications' is the name of a BibTeX file

%-----       letter       ---------------------------------------------------------

\end{document}


%% end of file `template.tex'.
