
%% start of file `template.tex'.
%% Copyright 2006-2013 Xavier Danaux (xdanaux@gmail.com).
%
% This work may be distributed and/or modified under the
% conditions of the LaTeX Project Public License version 1.3c,
% available at http://www.latex-project.org/lppl/.


\documentclass[11pt,a4paper,sans]{moderncv}        % possible options include font size ('10pt', '11pt' and '12pt'), paper size ('a4paper', 'letterpaper', 'a5paper', 'legalpaper', 'executivepaper' and 'landscape') and font family ('sans' and 'roman')

% modern themes
\moderncvstyle{banking}                            % style options are 'casual' (default), 'classic', 'oldstyle' and 'banking'
\moderncvcolor{black}                                % color options 'blue' (default), 'orange', 'green', 'red', 'purple', 'grey' and 'black'
%\renewcommand{\familydefault}{\sfdefault}         % to set the default font; use '\sfdefault' for the default sans serif font, '\rmdefault' for the default roman one, or any tex font name
%\nopagenumbers{}                                  % uncomment to suppress automatic page numbering for CVs longer than one page

% character encoding
\usepackage[utf8]{inputenc}                       % if you are not using xelatex ou lualatex, replace by the encoding you are using
%\usepackage{CJKutf8}                              % if you need to use CJK to typeset your resume in Chinese, Japanese or Korean

% adjust the page margins
\usepackage{geometry}
\geometry{a4paper, portrait, margin=0.5in}
%\setlength{\hintscolumnwidth}{3cm}                % if you want to change the width of the column with the dates
%\setlength{\makecvtitlenamewidth}{10cm}           % for the 'classic' style, if you want to force the width allocated to your name and avoid line breaks. be careful though, the length is normally calculated to avoid any overlap with your personal info; use this at your own typographical risks...

\usepackage{import}

% allows multi-line comments
\usepackage{verbatim}

\urlstyle{same}

% setting certain values for tabular environment to work
\setlength\arrayrulewidth{.4pt}
\setlength\tabcolsep{6pt}

% personal data
\name{Fawaz}{Shah}
%\title{Curriculum Vitae}                               % optional, remove / comment the line if not wanted
%\address{4 Cookham Dene Close, Chislehurst, Kent, BR7 5QW}% optional, remove / comment the line if not wanted; the "postcode city" and and "country" arguments can be omitted or provided empty
%\phone[mobile]{+44 7548822935}                   % optional, remove / comment the line if not wanted
%\phone[fixed]{01234 123456}                    % optional, remove / comment the line if not wanted
%\phone[fax]{+3~(456)~789~012}                      % optional, remove / comment the line if not wanted
\email{fs2217@ic.ac.uk}                               % optional, remove / comment the line if not wanted
%\homepage{fawazshah.github.io}                         % optional, remove / comment the line if not wanted
%\extrainfo{Joint Maths \& Computing (JMC) student}                 % optional, remove / comment the line if not wanted
%\photo[64pt][0.4pt]{picture}                       % optional, remove / comment the line if not wanted; '64pt' is the height the picture must be resized to, 0.4pt is the thickness of the frame around it (put it to 0pt for no frame) and 'picture' is the name of the picture file
%\quote{Some quote}                                 % optional, remove / comment the line if not wanted

% to show numerical labels in the bibliography (default is to show no labels); only useful if you make citations in your resume
%\makeatletter
%\renewcommand*{\bibliographyitemlabel}{\@biblabel{\arabic{enumiv}}}
%\makeatother
%\renewcommand*{\bibliographyitemlabel}{[\arabic{enumiv}]}% CONSIDER REPLACING THE ABOVE BY THIS

% bibliography with mutiple entries
%\usepackage{multibib}
%\newcites{book,misc}{{Books},{Others}}
%----------------------------------------------------------------------------------
%            content
%----------------------------------------------------------------------------------
\begin{document}
%\begin{CJK*}{UTF8}{gbsn}                          % to typeset your resume in Chinese using CJK
%-----       resume       ---------------------------------------------------------
\maketitle

\vspace{-40pt}

\begin{center}
\textbf{Personal Website}: \href{https://fawazshah.github.io/}{fawazshah.github.io} \\
\textbf{GitHub}: \href{https://github.com/fawazshah/}{github.com/fawazshah}
\end{center}

\section{Education}

\vspace{5pt}

\begin{itemize}

\item{
\cventry
{2017--2021}
{MEng. Mathematics and Computer Science (Joint Honours)}
{Imperial College London}
{}{}{
\noindent
\\
On track for First Class Honours
\\
Modules of interest: Java, Databases, Compilers, Algorithms, Statistical Modelling, Numerical Analysis
}
} % arguments 3 to 6 can be left empty

\item{
\cventry
{2010--2017}
{GCSEs and A Levels}
{St. Olave's Grammar School}
{}{}{
\noindent
\\
A Levels: 4 A*s (Maths, Further Maths, Chemistry, Physics)
}}

\end{itemize}

\section{Experience}

\begin{itemize}

\item{
\cventry
{Jul. - Sep. 2019}
{Software Engineering Intern}
{Bloomberg L.P.}
{}{}{
\begin{itemize}
\item Worked as part of Strategy Indices, a division who produce custom multi-asset indices implementing various trading strategies for enterprise clients
\item Developed a visualisation tool to display the structure, configuration and historical data of indices created with "Behaviours", a new internal C\# framework
\item Created user interfaces using WPF and XAML, with an SQL database behind the scenes
\end{itemize}
}}

\item{
\cventry
{Jul. - Aug. 2018}
{Internet Services Developer - Intern}
{Netcraft Ltd.}
{}{}{
\begin{itemize}
\item Worked on a feature allowing malicious URLs to be automatically taken down based on certificate authority
\item Refactored old Perl and SQL codebases for ease of use
\item Rewrote documentation for 20+ year old codebases with instructions for integrating with newer systems
\end{itemize}
}}

\item{
\cventry
{Apr. 2018 - present}
{Volunteer Web Developer}
{SolidariTee National}
{}{}{
\begin{itemize}
\item Redesigned \href{https://www.solidaritee.org.uk/}{solidaritee.org.uk} with Foundation to make it more mobile-friendly
\item Built a bespoke online shop, using a LAMP stack running Magento and hosted on DigitalOcean
\item Shop is online at \href{https://shop.solidaritee.org.uk/}{shop.solidaritee.org.uk}, has helped raise over £2500 for charity
\end{itemize}
}}

\end{itemize}

\section{Projects}

\begin{itemize}

\item{
\cventry
{Jun. 2019}
{University WebApps Project}
{FoodScape}
{}{}{
\begin{itemize}
\item FoodScape is a cross-platform mobile app designed to help students living in halls share food and reduce food waste
\item Developed using a Node.js + Express backend, Sequelize as an ORM, and Flutter for the front-end
\item Achieved 84\%, one of the highest grades in the year 
\end{itemize}
}}

\item{
\cventry
{Mar. 2019}
{Facebook Hack-A-Project}
{Studify}
{}{}{
\begin{itemize}
\item Created Studify, a multi-user website to help students connect and create revision study groups
\item Node.js backend, with a Firebase data store
\end{itemize}
}}

\end{itemize}

\section{Positions of Responsibility}

\begin{itemize}

\item{
\cventry
{Aug. 2018 - July 2019}
{Marketing \& Publicity Officer}
{DoCSoc (Imperial College Department of Computing Society)}
{}{}{
\begin{itemize}
\item Presented an introductory lecture on using Linux and the command line, attracting over 120 students
\item Negotiated over £5000 worth of society sponsorship with leading companies in the technology sector
\item In charge of social media and designing DoCSoc apparel
\end{itemize}
}}

\end{itemize}

\section{Notable Achievements}

\begin{itemize}

\item Awarded Union Colours for outstanding service to Imperial College Union

\item Awarded a Merit in Imperial College's Japanese Horizons Level 1 Course

\end{itemize}

% Publications from a BibTeX file without multibib
%  for numerical labels: \renewcommand{\bibliographyitemlabel}{\@biblabel{\arabic{enumiv}}}% CONSIDER MERGING WITH PREAMBLE PART
%  to redefine the heading string ("Publications"): \renewcommand{\refname}{Articles}
\nocite{*}
\bibliographystyle{plain}
\bibliography{publications}                        % 'publications' is the name of a BibTeX file

% Publications from a BibTeX file using the multibib package
%\section{Publications}
%\nocitebook{book1,book2}
%\bibliographystylebook{plain}
%\bibliographybook{publications}                   % 'publications' is the name of a BibTeX file
%\nocitemisc{misc1,misc2,misc3}
%\bibliographystylemisc{plain}
%\bibliographymisc{publications}                   % 'publications' is the name of a BibTeX file

%-----       letter       ---------------------------------------------------------

\end{document}


%% end of file `template.tex'.
